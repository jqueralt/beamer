% Basat en: /cvsroot/latex-beamer/latex-beamer/solutions/conference-talks/conference-ornate-20min.en.tex,v 1.6 2004/10/07 20:53:08 tantau Exp $
% Copyright 2004 by Till Tantau <tantau@users.sourceforge.net>.
% In principle, this file can be redistributed and/or modified under
% the terms of the GNU Public License, version 2.
%
% However, this file is supposed to be a template to be modified
% for your own needs. For this reason, if you use this file as a
% template and not specifically distribute it as part of a another
% package/program, I grant the extra permission to freely copy and
% modify this file as you see fit and even to delete this copyright
% notice. 

%modificat per Joan Queralt <jqueralt@gmail.com>
%==================
% definició de la classe de document
\documentclass{beamer}
% veure el manual de l'usuari per modificar el resultat final a <article> (per ser imprès en paper)
\mode<presentation>
{
%\usetheme{Boadilla}  
%\usetheme{Antibes}   %crea un índex en arbre a cada diapo
%\usetheme{JuanLesPins}

%\useinnertheme{inmargin} %posa els títols de bloc al marge esquerre
%\useoutertheme[hooks]{tree} %crea un índex en arbre a cada diapo
%\usefonttheme[stillsansserifmath]{serif} %lletra amb serif excepte les mates

\usepackage{keynote-vintage} %% http://www.ucl.ac.uk/~ucbpeal/latexposter.html
%\usecolortheme{beaver} 
  %\usetheme{Manhattan} % semblant a keynote
  %\usecolortheme{crane} %hi ha més colors per triar
  % per mostrar textos semiptransparents:
  \setbeamercovered{transparent} 
}
% ============= definició d'entorn teorema basat en l'original Theorem
\newtheorem{teorema}[theorem]{Teorema}

% ============= paquets de llenguatge 
\usepackage[catalan]{babel}
%\usepackage[latin1]{inputenc}
\usepackage[T1]{fontenc}
\usepackage[utf8]{inputenc}


% ============= paquet de tipus de lletra
%per defecte beamer usa Computer Modern
%\usepackage{times} 
\usepackage{bera}   %  amb i sans serif

\usepackage{amsmath}

% ============= dades del document
\title[Beamer]{Actualització de beamer}

\subtitle{Novetats}

\author[Joan Queralt]{Joan Queralt Gil}

\institute[cata\LaTeX] 
{
  cata\LaTeX{}\\
  El lloc català de \LaTeX{}
}

\date[30.07.2012]{\today}

\titlegraphic{\includegraphics[scale=.1]{qrplanet}}


\subject{Tema de la presentació} % Només apareixerà a les propietats del document PDF

% =============== final del preàmbul

\begin{document}

% ++++++++ 1er marc amb el títol
\begin{frame}
  \titlepage
  \transblindshorizontal
\end{frame}

% ++++++++ 2on marc amb la Taula de continguts TDC

\begin{frame}
  \frametitle{Taula de continguts} %Títol del marc
  \framesubtitle{transblindsvertical}
  
  \tableofcontents[pausesections] % TDC
  \transblindsvertical
   

  % You might wish to add the option [pausesections]
\end{frame}

% +++++++++ Aquí comença la presentació
\section{Marcs} % les seccions i subseccions són fora dels marcs
\subsection{Enumeracions}

% +++++++++ 3er marc

\begin{frame}
  \frametitle{Títol del marc}
  \framesubtitle{transboxin}
   \begin{itemize}
		  \item 
		    un punt
		  \item
		    un \texttt{altre} punt
		  \item
		  un tercer punt
  \end{itemize}
  \transboxin
  

\end{frame}

% +++++++++ 4rt marc que genera vàries diapositives o pàgines al PDF
\begin{frame}
  \frametitle{Un marc = n diapositives}
  \framesubtitle{transboxout}

  Un sol marc pot generar diferents diapositives\dots
  \begin{itemize}
  \item usant el comandament \texttt{pause}:
    \begin{itemize}
    \item
      primer punt
      \pause
    \item    
      segon punt
    \end{itemize}
  \item
    usant especificacions de diapositiva:
    \begin{itemize}
    \item<3->
      punt no.1
    \item<4->
      punt no.2
    \end{itemize}
  \item
    usant el commandament \texttt{uncover}
    \begin{itemize}
      \uncover<5->{\item
        punt primer}
      \uncover<6->{\item
        punt segon}
    \end{itemize}
  \end{itemize}
  \transboxout
  
\end{frame}

\subsection{Columnes}
% +++++++++  marc
\begin{frame}
  \frametitle{Marc amb columns}
  \framesubtitle{transdissolve[duration=0.5]}
  \begin{columns}
    \column{.5\textwidth}
			  \begin{itemize}
				\item punt 1
				\item punt 2
			  \end{itemize}
    \column{.5\textwidth}
			  \begin{enumerate}
				\item punt 1
				\item punt 2
			  \end{enumerate}
  \end{columns} 
  \transdissolve[duration=0.5]
  
\end{frame}
% +++++++++  marc
\begin{frame}
  \frametitle{Un altre marc amb columns i block}
  \framesubtitle{transglitter[direction=90]}
  \begin{columns}
    \column{.45\textwidth}
			    \begin{block}{Títol del bloc}
				\begin{itemize}
					\item punt primer
					\item punt segon
				\end{itemize}
\end{block}
    \column{.45\textwidth}
			    \begin{block}{Títol del bloc}
				\begin{itemize}
					\item punt 1
					\item punt 2
				\end{itemize}
\end{block}
  \end{columns} 
  \transglitter[direction=90]
  

\end{frame}
% +++++++++  marc
\section{Entorns}
\subsection{Blocs}
% +++++++++  marc
\begin{frame}
  \frametitle{Marc amb entorn block}
  \framesubtitle{transsplitverticalin}
  \begin{block}{Títol del bloc}
				\begin{itemize}
					\item punt primer
					\item punt segon
				\end{itemize}
	\end{block} 

	\transsplitverticalin
	
\end{frame}
% +++++++++  marc
\begin{frame}
  \frametitle{Marc amb entorn Teorema}
  \framesubtitle{transsplitverticalout}
  \begin{teorema}{Teorema 1}
				\begin{itemize}
					\item Tal
					\item Qual
				\end{itemize}
\end{teorema} 
 	\transsplitverticalout
	

\end{frame}
% +++++++++  marc
\begin{frame}
  \frametitle{Marc amb entorn Example}
  \framesubtitle{transwipe[direction=90]}

\begin{exampleblock}{Exemple}
Exemple de bloc d'exemple:
   \begin{itemize}
	\item Tal
	\item Qual
   \end{itemize}
\end{exampleblock}

\transwipe[direction=90]

\end{frame}
% +++++++++  marc
\begin{frame}
  \frametitle{Marc amb entorn Proof}
    \framesubtitle{transblindshorizontal}

  \begin{proof}{Prova 1}
				\begin{itemize}
					\item Tal
					\item Qual
				\end{itemize}
\end{proof} 
\transblindshorizontal
\end{frame}


% +++++++++  marc
\begin{frame}
  \frametitle{Marc amb entorn Alert}
    \framesubtitle{transblindshorizontal}

      \begin{alertblock}{Atenció}
           $H_2 SO_4 + Zn \rightarrow ZnSO_4 + H_2\uparrow $
      \end{alertblock}
\transblindshorizontal
\end{frame}

% +++++++++  marc
\begin{frame}
  \frametitle{Marcs i quadres de text}
    \framesubtitle{Un post-it}
      \setbeamercolor{postit}{fg=black,bg=yellow}  
      \begin{beamercolorbox}[sep=1em,wd=5cm,shadow=true,rounded=true]{postit}
            \centering            
            Recorda-te'n!!
      \end{beamercolorbox}
\end{frame}
     
% +++++++++  marc
\begin{frame}
  \frametitle{Marcs i quadres de text 2}
    \framesubtitle{Un text emmarcat}     
      \begin{beamerboxesrounded}[upper=block body,lower=block body,shadow=true,width=6cm]{Caixa arrodonida de 6 cm}
       \centering
             $3 H_2 + N_2 \rightarrow 2 NH_3$
      \end{beamerboxesrounded}

  \transwipe[direction=90]
\end{frame}


\subsection{Botons}
% +++++++++  marc
\begin{frame}
  \frametitle{Botons}
    \framesubtitle{transwipe[direction=-45]}
    \begin{block}{Botons amb enllaços}
           \begin{itemize}
					\item \href{http://phobos.xtec.cat/jqueralt}{\beamerbutton{cata\LaTeX{}}}
		            \item \hyperlinkpresentationstart{\beamerbutton{Inici}}
		            \item \hyperlinkpresentationend{\beamerbutton{Final}}
		            \item \hyperlinksectionstart{\beamerbutton{Inici de secció}}
		            \item \hyperlinksectionend{\beamerbutton{Final de secció}}
		   \end{itemize}
	\end{block} 	
  \transwipe[direction=-45]
\end{frame}







\section{Aplicacions}
% +++++++++  marc
\begin{frame}
  \frametitle{Matemàtiques}
    \framesubtitle{transwipe[direction=45]}
  
    \begin{block}{Equació de segon grau completa}
    \begin{align*}
    a \cdot x^{2}+ b \cdot x + c &= 0\\
  	x &=\frac{-b\pm\sqrt{b^{2}-4 \cdot a \cdot c}}{2 \cdot a}
    \end{align*}
  \end{block}
  \transwipe[direction=45]
\end{frame}
% +++++++++  marc
\begin{frame}
  \frametitle{Química}
  \framesubtitle{transduration{2}}
  \begin{block}{Obtenció del sulfat amònic}
 	\[H_{2}SO_{4} + 2 \cdot NH_{3} \rightarrow (NH_{4})_{2}SO_{4}
\]
\end{block}
\transduration{2}
\end{frame}
% +++++++++  marc
\begin{frame}
  \frametitle{Física}
  \framesubtitle{transwipe[direction=270]}
  \begin{block}{Força}
  	\[\vec{F} = m \cdot \vec{a}
  	\]
  	\end{block}
  	\transwipe[direction=270]
\end{frame}

\section*{Resum}
% +++++++++  marc
\begin{frame}
  \frametitle<presentation>{Resum}
  \framesubtitle{transwipe[direction=90]}
  \begin{itemize}
			  \item
			    La \alert{primera idea} important.
			    \pause
			  \item
			    La \alert{segona idea} important.
			    \pause
			  \item
			    La \alert{tercera idea} important.
  \end{itemize} 
  \transwipe[direction=90]
\end{frame}

% =============== opcional: apèndix o bibliografia
\appendix
\section<presentation>*{\appendixname}
\subsection<presentation>*{Crèdits}
% +++++++++  marc
\begin{frame}
  \frametitle<presentation>{Fet amb}
  \framesubtitle{transdissolve[duration=0.5]}
   Fet amb \alert{programari lliure}: \textbf{beamer} de \LaTeX{} 
   \transdissolve[duration=0.5]
\end{frame}

\end{document}
